\chapter{Goals and Future Work}

\subsection{Goals for Thesis}

\begin{enumerate}
	\item Describe the basics cryptography 
	\item Describe the nature of dynamical systems and chaos
	\item Describe the different types of chaotic encryption and how the key parameters vary
	\item Propose improvements for existing chaotic cryptosystems
\end{enumerate}

\subsection{Future Work}

I first became interested in chaotic encryption when I stumbled across a side-note in \textit{Nonlinear Dynamics and Chaos} by Steven Strogatz describing how one of his students had build a physical circuit that used the principles of chaos to encrypt then decrypt input signals.  Before I had a chance to tinker with circuits myself, though, I got sidetracked in all of the other applications of chaos, namely how its an emerging field of encryption, largely of interest to computer science researchers.  For further studies, though, I'd like to take the ideas and methods documented in this thesis and apply them to physical objects, such as building synchronised systems in a laboratory.  There is still a lot of be researched in making these physical systems more efficient and more easily adjustable to new keys and parameters.  I am also interested in testing these systems in various environments--does natural interference get filtered out in the encryption process, or can it obstruct decryption entirely?